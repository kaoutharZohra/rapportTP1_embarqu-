\documentclass[12pt,a4paper]{article}
\usepackage[utf8]{inputenc}
\usepackage{natbib}
\usepackage{graphicx}
\usepackage{float}
\setcounter{secnumdepth}{5}
\setcounter{tocdepth}{4}
\usepackage{tabularx}

\usepackage{multirow}
\usepackage{framed}
\usepackage{listings}
\usepackage{lmodern}

\usepackage{minted}
\usepackage{color}
\begin{document}
\begin{center}

 
  
 
   \large
  	\vspace{2cm}
  \textbf{Université Paris Diderot- Master 2}\\
  \vspace{1cm}
  \LARGE
  \textbf{Programmation embarquée}\\
   \vspace*{1cm}
\LARGE
  \textbf{TP N°1}
 
  \large
  \LARGE
  \setlength{\fboxsep}{0.5cm}
  \begin{framed}
	\textbf{Langages, techniques et outils}
  \end{framed}
  \vspace{2cm}
\begin{table}[H]
   \setlength{\tabcolsep}{2cm}
    \large
	\centering
	\begin{tabular}{l}
		\textbf{Réalisé par :}    
		 \\  \\
		 -\textbf{???} ???\\
		- \textbf{ KEBAILI } Zohra Kaouter\\
	
	
		
  

	\end{tabular}
  \end{table}
  \vspace{\fill}
  \large
  \textbf{Promotion 2018/2019}
   \end{center}
\newpage
\section{Développez le générateur de données}

\begin{minted}{c}
#include <stdio.h>
#include <stdlib.h>
#include <stdint.h>
#include "header.h"
/**
* Remplis un tableau avec la suite de Fibonnacci.
* @param output_array tableau où la suite est enregistré
* @param size nombre de valeur à inscrire dans \a output_array
* @param min_value valeur jusqu'à laquelle les nombres ne sont pas enregistrés

*/
void fibonacci(uint32_t output_array[], uint32_t size, uint32_t min_value)
{
	uint32_t precprec=0,prec=1,c=0,i,tmp;
	if(size>0)
	{
	for(i=0;i<size;i++)
	{
		if(i>1)
		{
			tmp=precprec+prec;
			precprec=prec;
			prec=tmp;
		}
		else
		{
			tmp=i;

		}
		if(tmp> min_value)
		{
			output_array[c]=tmp;
			c++;
		}
	}
	}
}

\end{minted}



\end{document}
